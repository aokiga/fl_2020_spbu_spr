\documentclass[12pt]{article}
\usepackage[left=2cm,right=2cm,top=2cm,bottom=2cm,bindingoffset=0cm]{geometry}
\usepackage[utf8x]{inputenc}
\usepackage[english,russian]{babel}
\usepackage{cmap}
\usepackage{amssymb}
\usepackage{amsmath}
\usepackage{url}
\usepackage{pifont}
\usepackage{tikz}
\usepackage{verbatim}

\usetikzlibrary{shapes,arrows}
\usetikzlibrary{positioning,automata}
\tikzset{every state/.style={minimum size=0.2cm},
initial text={}
}


\newenvironment{myauto}[1][3]
{
  \begin{center}
    \begin{tikzpicture}[> = stealth,node distance=#1cm, on grid, very thick]
}
{
    \end{tikzpicture}
  \end{center}
}


\begin{document}
\begin{center} {\LARGE Формальные языки} \end{center}

\begin{center} \Large домашнее задание до 23:59 14.05 \end{center}
\bigskip

\begin{enumerate}
  \item[\bf 2)] $S \to aSbbbb\ |\ aaaSbb\ |\ c$\\
  Эта грамматика задает язык $L = \{a^{n + 3m}cb^{4n+2m}\}$\\
  Заметим, что порядок применения правил не важен. Тогда можем сделать эквивалентую грамматику, в которой будем сначала применять первое правило n раз, а потом второе m раз.\\
  $S_1 \to aS_1bbbb\ |\ S_2\\$
  $S_2 \to aaaS_2bb\ |\ c$\\
  Заметим, это - то что нам нужно. Докажем, что можем однозначно восстановить.\\
  Последнее правилом было $S_2 \to c$. Пусть у нас $C_a$ букв 'a' и $C_b$ букв 'b'. Тогда, мы можем найти n и m, решив систему:\\
  $\begin{cases}
  	n + 3m = C_a\\
  	4n + 2m = C_b
  \end{cases} \Leftrightarrow \begin{cases}
  10m = 4C_a - C_b\\
  20n = 6C_b - 4С_a\\
  \end{cases} \Leftrightarrow \begin{cases}
  m = \frac{4C_a - C_b}{10}\\
  n = \frac{3C_b - 2C_a}{10}\\
  \end{cases}$\\
  Можем однозначно восстановить n и m, значит можем грамматика однозначная.
  \item[\bf 3)] $F \to \varepsilon\ |\ aFaFbF$ \\
  Назовем это ПСП'\\
  Это очень похоже на обычное ПСП, только вместо одной '(' - у нас будет стоять две и между этими двумя скобками можно запихнуть такое же ПСП'.\\
  Условия принадлежности слова языку:\\
  $ka_s$ - количество букв 'a' в s, $kb_s$ - количество букв 'b' в s\\
  1) $ka_s - 2kb_s = 0$\\ 
  2) $\forall$ префикса p выполняется, что $ka_p - 2kb_p \ge 0$
  
  \item[\bf 4)] Пусть $L_1$ - язык, порожденной F, а $L_2$ - язык, порожденный $L_2$. 
  Сразу заметим, что слова из $L_2$ - слова нечетной длины, в которых на нечетных местах 'a' | 'c', а на четных 'a' | 'b'\\ 
  Посмотрим какие слова из $L_1$ принадлежат пересечению. При выводе этих слов, мы не можем 2 раза подряд юзать правила 2 и 3, потому что тогда слова не будут принадлежать языку, порожденного второй грамматикой.\\
  Слова пересечения будут принадлежать такой грамматике L:\\
  $A \to a\ |\ cDD\ |\ cCA$ \\
  $B \to cDA$ \\
  $C \to a\ |\ bB$ \\
  $D \to bA$ \\
  Что это вообще за штука?\\
  A - начинается с 'a' или 'c', нечетной длины.\\
  B - начинается с 'a' или 'c', четной длины.\\
  C - начинается с 'a' или 'b', нечетной длины.\\
  D - начинается с 'a' или 'b', четной длины.\\
  Слова из L очевидно лежат в $L_1$. Слова из L лежат в $L_2$, потому что на каждом этапе мы меняем четность и все корректно.
 
\end{enumerate}
\end{document}