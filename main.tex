\documentclass[12pt]{article}
\usepackage[left=2cm,right=2cm,top=2cm,bottom=2cm,bindingoffset=0cm]{geometry}
\usepackage[utf8x]{inputenc}
\usepackage[english,russian]{babel}
\usepackage{cmap}
\usepackage{amssymb}
\usepackage{amsmath}
\usepackage{url}
\usepackage{pifont}
\usepackage{tikz}
\usepackage{verbatim}

\usetikzlibrary{shapes,arrows}
\usetikzlibrary{positioning,automata}
\tikzset{every state/.style={minimum size=0.2cm},
initial text={}
}


\newenvironment{myauto}[1][3]
{
  \begin{center}
    \begin{tikzpicture}[> = stealth,node distance=#1cm, on grid, very thick]
}
{
    \end{tikzpicture}
  \end{center}
}


\begin{document}
\begin{center} {\LARGE Формальные языки} \end{center}

\begin{center} \Large домашнее задание до 23:59 26.03 \end{center}
\bigskip

\begin{enumerate}
  \item [\bf \textnumero 2] Приведите грамматику в нормальную форму Хомского: \\
  $S\ \to\ RS\ |\ R\\
   R\ \to aSb\ |\ cRd\ |\ ab\ |\ cd\ |\ \varepsilon$\\
   Терминалы: $a,\ b,\ c,\ d$, нетерминалы: $R,\ S$, стартовый нетерминал: $S$, пустая строка: $\varepsilon$.\\
   {\bf Решение}\\
   Поменяем старт. Получим:\\
   $S \to S'\\
   S' \to RS'\ |\ R\\
   R \to aS'b\ |\ cRd\ |\ ab\ |\ cd\ |\ \varepsilon$
   
   Устраним длинные правила. Получим:\\
   $S \to S'\\
   S' \to RS'\ |\ R\\
   R \to EB\ |\ FD\ |\ AB\ |\ CD\ |\ \varepsilon\\
   A \to a\\
   B \to b\\
   C \to c\\
   D \to d\\
   E \to AS'\\
   F \to CR$
   
   Устраним $\varepsilon$-правила. Получим:\\
   $S \to S'\ | \varepsilon\\
   S' \to RS'\ |\ R\\
   R \to EB\ |\ FD\ |\ AB\ |\ CD\\\
   A \to a\\
   B \to b\\
   C \to c\\
   D \to d\\
   E \to AS'\ |\ A \\
   F \to CR\ |\ C$
   
   Устраним цепные правила. Получим:\\
   $S \to RS'\ |\ EB\ |\ FD\ |\ AB\ |\ CD\ | \varepsilon\\
   S' \to RS'\ |\ EB\ |\ FD\ |\ AB\ |\ CD\\
   R \to EB\ |\ FD\ |\ AB\ |\ CD\\\
   A \to a\\
   B \to b\\
   C \to c\\
   D \to d\\
   E \to AS'\ |\ a \\
   F \to CR\ |\ c$\\
   Эта грамматика в нормальной форме Хомского.
  \item [\bf \textnumero 3] Является ли следующий язык контекстно-свободным? Если является, привести КС грамматику, иначе -- доказать. \\
  $L = \{\ a^m\ b^n\ |\ m\ +\ n\ >\ 0,\ (m\ +\ n)\ \vdots\ 2\ \}$ \\
  {\bf Решение}\\
  Сделаем такую КС грамматику: $S \to aa\ |\ bb\ |\ ab\ |\ aaS\ |\ aSb\ |\ Sbb$.\\ Пусть она задает язык L'.\\
  Нетрудно понять, что все слова в L' будут непустые и четной длины.\\
  Докажем, что L = L'.\\
  Сначала, включение в одну сторону: L $\subset$ L'. \\
  Нетрудно понять, что все слова из букв a и все слова из букв b, которые лежат в L, также лежат и в L' (по индукции).\\
  Рассмотрим слово w = $a^kb^m$, w $\in$ L. Пусть k $\le$ m.\\
  Удалим из слова $\lfloor\frac{k}{2}\rfloor$ строк "аa" сначала. Если k нечетное, то после этого удалим букву "a"  с начала и букву "b" с конца. Получим слово w', состоящее только из букв b. Оно лежит в L и L' (так как состоит только из букв b).\\
  Мы можем по правилам получить слово w из слова w'. Значит, w $\in$ L'.\\
  Если k > m то сделаем аналогично, только вместо одного правила будем юзать другое и получим слово, состоящее только из букв "a". 
  Включение в другую сторону - очевидно. Все слова из L' подходят в качестве слов из L. Так как они четной длины, непустые и имеют вид $a^kb^m$.
\end{enumerate}
\end{document}
